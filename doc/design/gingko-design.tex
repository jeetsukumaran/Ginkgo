\documentclass[11pt]{article}
\usepackage{geometry}                % See geometry.pdf to learn the layout options. There are lots.
\geometry{letterpaper}                   % ... or a4paper or a5paper or ... 
%\geometry{landscape}                % Activate for for rotated page geometry
%\usepackage[parfill]{parskip}    % Activate to begin paragraphs with an empty line rather than an indent
\usepackage{graphicx}
\usepackage{amssymb}
\usepackage{epstopdf}
\usepackage{natbib}


\usepackage{url}
\usepackage{algorithmic}
\usepackage{algorithm}
\usepackage{listings}

\DeclareGraphicsRule{.tif}{png}{.png}{`convert #1 `dirname #1`/`basename #1 .tif`.png}


%%%%%%%%%%%%%%%%%%%%%%%%%%%%%%%%%%%%%%%%%%%%%%%%%%%%
%% Names, etc.


%%%%%%%%%%%%%%%%%%%%%%%%%%%%%%%%%%%%%%%%%%%%%%%%%%%%
%% Document meta-data 
\title{Gingko: Forward-Time Simulation of Genealogical Trees Under Spatially and Environmentally Dynamic Biogeographical Histories}
\author{Jeet Sukumaran}
\date{} 


\begin{document}
\maketitle

\section{Introduction and Overview}

\subsection{Objectives}

GINGKO will simulate genealogical trees for multiple locii in populations of organisms evolving and interacting across a virtual landscape.
The landscape will incorporate both spatial and environmental structuring, with the spatial relationships between discrete locations (\textit{cells}) and the environmental characteristics of those locations, changing over time to simulate various biogeographical scenarios or histories.
The spatial relationships between cells will effect the migration of organisms across the landscape, while the environmental characteristics of cells  will effect the survival probabilities of organisms occupying those cells.

The trees generated by GINGKO, or DNA sequences simulated on those trees, will be used to answer the following questions:

\begin{itemize}
	\item What is the effect of spatial structuring (barriers to migration) on divergence time and population size estimation methods, such as BEAST?
	\item What are the false positive and false negative error rates of Nested Clade Analysis given various phylogeographic histories?
\end{itemize}

%\subsection{Inputs}
%GINGKO will use ESRI ARC/INFO ASCII GRID (\url{http://en.wikipedia.org/wiki/ESRI_grid}) to specify the landscape. 




\subsection{Components}

The full GINGKO pipeline will consist of three components:

\begin{enumerate}
\item A primary simulator program, which will carry out a single replicate of a forward-time simulation under a particular biogeographical history, and produce a file of ancestor identity histories for each neutral locus tracked in each species. 
\begin{itemize}
\item The GINGKO library will provide for landscape spatial and environmental data to be loaded dynamically from external ESRI ARC/INFO ASCII GRID files. A schedule file will specify the suites of files to be loaded at particular generations to effect changes in spatial and/or environmental conditions.
\item Organism classes in the GINGKO libraries will be specialized by client code specific to each simulation scenario, to control details such as mating and movement ecologies, fitness functions, etc..
\item Thus, GINGKO will consist of multiple simulation programs, with each simulation program specialized in the particulars of the ecologies of its organismal agents, while landscapes and landscape histories are specified through different external input data files (landscape schedules and the ASCII GRID files).
\end{itemize}
\item A tree compilation script, which will take the ancestor identity history and compile a phylogenetic tree relating each allele.
\item A sequence generation script, which will generate DNA or protein character sequence data under various finite-state models on trees.
\end{enumerate}




\section{Simulation of Ancestor Identities Histories}

\begin{itemize}

%\item The primary simulation program will typically be customized for each biogeographical scenario, extending base classes provided in the GINGKO library to emulate particular mating or movement ecologies.
\item An ancestor identity history for alleles in a particular locus is simply a listing of ancestors for each allele in that locus in at a particular snapshot in time. Alleles are identified by an arbitrary number, guaranteed to be unique across all alleles within a particular locus across the entire lifespan of the simulation.



\end{itemize}


\end{document}  



%%%%%%%%%%%%%%%%%%%%%%%%%%%%%%%%%%%%%%%%%%%%%%%%%%%%%%%%%%
%%%%%%%%%%% SCRATCH %%%%%%%%%%%%%%%%%%%%%%%%%%%%%%%%%%%%%%%%%

\lstset{language=Python}          % Set your language (you can change the language for each code-block optionally)
 \begin{lstlisting}
#! /bin/usr/env python
import sys

class K(object):
	def __init__(self, s):
		self.s = 1

if __name__ == "__main__":
	print "hello, world"		
\end{lstlisting}

\lstset{language=C++} 
\begin{lstlisting}
#import <iostream>

int main(argc, argv* []) {
	std::cout << "Hello, world" << std::endl;
}	
\end{lstlisting}