\documentclass[11pt]{article}
\usepackage{geometry}                % See geometry.pdf to learn the layout options. There are lots.
\geometry{letterpaper}                   % ... or a4paper or a5paper or ... 
%\geometry{landscape}                % Activate for for rotated page geometry
%\usepackage[parfill]{parskip}    % Activate to begin paragraphs with an empty line rather than an indent
\usepackage{graphicx}
\usepackage{amssymb}
\usepackage{epstopdf}
\usepackage{natbib}
\DeclareGraphicsRule{.tif}{png}{.png}{`convert #1 `dirname #1`/`basename #1 .tif`.png}

\title{Gingko: Forward-Time Simulation of Genealogical Trees Under Spatially and Environmentally Dynamic Biogeographical Histories}
\author{Jeet Sukumaran}
\date{} 


\begin{document}
\maketitle
\section{Introduction}

GINGKO will simulate genealogical trees for multiple locii in populations of organisms evolving and interacting across a virtual landscape.
The landscape will incorporate both spatial and environmental structuring, with the spatial relationships between discrete locations (\textit{cells}) and the environmental characteristics of those locations, changing over time to simulate various biogeographical scenarios or histories.
The spatial relationships between cells will effect the migration of organisms across the landscape, while the environmental characteristics of cells  will effect the survival probabilities of organisms occupying those cells.

The trees generated by GINGKO, or DNA sequences simulated on those trees, will be used to answer the following questions:

\begin{enumerate}
	\item What is the effect of spatial structuring (barriers to migration) on divergence time and population size estimation methods, such as BEAST?
	\item What are the false positive and false negative error rates of Nested Clade Analysis given various phylogeographic histories?
\end{enumerate}


%\subsection{}



\end{document}  